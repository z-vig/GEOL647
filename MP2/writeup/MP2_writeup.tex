\documentclass{article} % This command is used to set the type of document you are working on such as an article, book, or presenation

\usepackage{geometry} % This package allows the editing of the page layout
\geometry{top=1in,bottom=1in,left=1in,right=1in,a4paper}
\usepackage{float}
\usepackage{amsmath}  % This package allows the use of a large range of mathematical formula, commands, and symbols
\usepackage{amssymb}
\usepackage{graphicx}  % This package allows the importing of images
\usepackage{gensymb}
\usepackage{enumitem}
\usepackage{booktabs}
\usepackage{siunitx}
\usepackage{bm}
\usepackage{pdfpages}
\usepackage{wrapfig}

\newcommand{\question}[2][]{\begin{flushleft}
        \textbf{Problem #1}: #2
\end{flushleft}}

\newcommand{\sol}{\textbf{Solution}:} %Use if you want a boldface solution line
\newcommand{\maketitletwo}[2][]{
        \begin{center}
        \Large{\textbf{Matlab Practical 2}
            
            GEOL647} % Name of course here
        \vspace{5pt}
        
        \normalsize{Zach Vig  % Your name here
        
        \today}        % Change to due date if preferred
        \vspace{15pt}
        
        \end{center}}

\def\doubleunderline#1{\underline{\underline{#1}}}

\begin{document}
\maketitletwo[2]

\question[1.1]{What is the analytic expression for the frequency response function of a simple seismometer with mass \textbf{m}, spring constant \textbf{k}, and coefficient of friction \textbf{D}?}

\sol In the Fourier domain, the ODE for seismometer motion is:
\begin{align*}
\omega^2\tilde{u}(\omega) = \frac{k}{m}\tilde{x}(\omega)+i\omega\frac{D}{m}\tilde{x}(\omega)-\omega^2\tilde{x}(\omega)
\end{align*}

Where D is the damping term, u is the relative position of the Earth (input) and x is the relative position of the seismometer (output). So the frequency response function (FRF) is:

\begin{align*}
        \tilde{X}(\omega)= \frac{\tilde{x}(\omega)}{\tilde{u}(\omega)} = \frac{\omega^2}{\frac{k}{m}+\frac{D}{m}i\omega-\omega^2}
\end{align*}

\question[1.2]{Derive the amplitude response function of the seismometer from the FRF:}

\sol
\begin{align*}
        ARF = \sqrt{|\tilde{X}(\omega)|} &= \sqrt{\frac{\omega^2}{\frac{k}{m}+\frac{D}{m}i\omega-\omega^2}\frac{\omega^2}{\frac{k}{m}-\frac{D}{m}i\omega-\omega^2}} \\
        ARF &= \frac{\omega^2}{\sqrt{\left(\frac{k}{m}-\omega^2\right)^2+\left(\frac{D}{m}\omega\right)^2}}
\end{align*}

\question[1.3]{Derive the phase delay function of the seismometer from the FRF:}

\sol
\begin{align*}
        \Theta (\omega) &= \arctan\left(\frac{Im(\tilde{X}(\omega))}{Re(\tilde{X}(\omega))}\right) \\
\end{align*}

To find this, we must rewrite $\tilde{X}(\omega)$ as:

\begin{align*}
        \tilde{X}(\omega) &= \frac{\omega^2}{\frac{k}{m}+\frac{D}{m}i\omega-\omega^2}\frac{\frac{k}{m}-\frac{D}{m}i\omega-\omega^2}{\frac{k}{m}-\frac{D}{m}i\omega-\omega^2} = \frac{-\omega^4+\omega^2\frac{k}{m}-i\omega^3\frac{D}{m}}{\left[\frac{k}{m}-\omega^2\right]^2+\left(\frac{D}{m}\omega\right)^2} \\ 
        Im(\tilde{X}(\omega)) &= \frac{-\omega^3\frac{D}{m}}{\left[\frac{k}{m}-\omega^2\right]^2+\left(\frac{D}{m}\omega\right)^2}, \quad Re(\tilde{X}(\omega)) = \frac{-\omega^4 + \omega^2\frac{k}{m}}{\left[\frac{k}{m}-\omega^2\right]^2+\left(\frac{D}{m}\omega\right)^2}
\end{align*}

Plugging this in, we have:

\begin{align*}
        \Theta(\omega) = \arctan\left(\frac{-\omega^3\frac{D}{m}}{-\omega^4+\omega^2\frac{k}{m}}\right) = \arctan\left(\frac{-\omega\frac{D}{m}}{\frac{k}{m}-\omega^2}\right) = \arctan\left(\frac{-\omega D}{k-m\omega^2}\right)
\end{align*}

\newpage
\question[2.1]{Write the frequency response function for a seismometer with this type of feedback system. To do this, think about inputs and outputs. Because voltage is output by the system and ground acceleration is input, compute $V/\ddot{u}$.}
\[(1)\text{\space}\ddot{y} = \ddot{u}-\ddot{x}\quad\quad (2)\text{\space}\ddot{x}=\beta V\quad\quad (3)\text{\space}V=\frac{KA\omega^2}{\omega^2-2i\epsilon\omega-\omega^2_0}\ddot{y}\]
\sol \space To begin, we will rewrite certain terms to make things clearer:

\begin{align*}
        V &= KA\tilde{X}_0\ddot{y}\\
        \ddot{u} &= \ddot{y}+\ddot{x} = \ddot{y}+\beta KA\tilde{X}_0\ddot{y}
\end{align*}

\noindent Where $\tilde{X}_0$ is the FRF of an inertial seismometer. Now, solving for the FRF, we have:
\begin{align*}
        \tilde{X}(\omega) = \frac{V}{\ddot{u}} &= \frac{KA\tilde{X}_0\ddot{y}}{\ddot{y}+\beta KA\tilde{X}_0\ddot{y}} \\
        \tilde{X}(\omega) &= \frac{KA\tilde{X}_0}{1+\beta KA\tilde{X}_0}\\
        \tilde{X}(\omega) &= \frac{KA\omega^2}{\omega^2-2i\epsilon\omega-\omega_0^2+\beta KA\omega^2}\\
        \tilde{X}(\omega) &= \frac{KA\omega^2}{(1+\beta KA)\omega^2 - 2i\epsilon\omega - \omega_0^2}
\end{align*}

\question[2.2]{Solve for the amplitude response function of the system (I'm going to call $ARF(\omega)$, $\xi(\omega)$ because I think it's cool):}
\sol

\begin{align*}
        \xi(\omega) &= |\tilde{X}(\omega)| \\
        \xi(\omega) &= \sqrt{\frac{KA\omega^2}{(1+\beta KA)\omega^2 - 2i\epsilon\omega - \omega_0^2}\frac{KA\omega^2}{(1+\beta KA)\omega^2 + 2i\epsilon\omega - \omega_0^2}} \\
        \xi(\omega) &= \frac{KA\omega^2}{\sqrt{\left[(1+\beta KA)\omega^2-\omega_0^2\right]^2+4\epsilon^2\omega^2}}
\end{align*}


\end{document}